\documentclass[a4paper,12pt]{scrartcl}
\usepackage[utf8]{inputenc}
\usepackage[ngerman]{babel}
\usepackage[T1]{fontenc}
\usepackage[left= 3.5cm,right = 3cm, bottom = 4 cm, top = 2.5cm]{geometry}
\usepackage[onehalfspacing]{setspace}
%\usepackage{here}
%\usepackage{float}


\usepackage[table]{xcolor}
\usepackage{xcolor}
\usepackage{listings}
%\usepackage{acronym}
%\usepackage{mcode/mcode}
% ============= Packages =============

% Dokumentinformationen



% Standard Packages
\usepackage{graphicx}
%\usepackage{subfigure}
\usepackage{fancyhdr}
\usepackage{lmodern}
%\usepackage{color}
%\usepackage{transparent}
%\usepackage{tabu}
%\usepackage{siunitx}
%\usepackage{here}
%\usepackage{tikz}
%\usepackage{caption} 
%\usepackage{slashed}
%\usepackage{cancel}
\usepackage{multicol}
\usepackage{tabularx}
\usepackage{glossaries}
%\usepackage{adjustbox}
%\usepackage{wallpaper}
%\usepackage{transparent}

%\usepackage[style=authortitle-icomp]{biblatex} 
%\bibliography{C:\Users\Gabriel\Desktop\Bachelorarbeit\Fachmodul\Bericht\literatur} 

%\sisetup{detect-weight=true, detect-family=true}

\graphicspath{{img/}}

% Aufzählung Packages
%\usepackage{paralist}

% Tabellen Packages
%\usepackage{booktabs}
\usepackage{multirow}
%\usepackage{cite}
%\usepackage{multibib}

% zusätzliche Schriftzeichen der American Mathematical Society
\usepackage{amsfonts}
\usepackage{amsmath}
%\usepackage{mathtools}
%Grad Zeichen
%\usepackage{textcomp}
%\usepackage{courier}

% nicht einrücken nach Absatz
\setlength{\parindent}{0pt}

% Literaturverzeichnis
\usepackage{url}

\usepackage{hyperref} % ermöglicht Hyperlinks in PDF-Dokumenten
\usepackage[ngerman]{cleveref}

% ========================== Kopf- und Fußzeile =======================================
\pagestyle{fancy}
%
\lhead{} %\thepage
%\chead{}
\rhead{\slshape \leftmark}
%%
\lfoot{}
\cfoot{\thepage}
\rfoot{}

%%
\renewcommand{\headrulewidth}{0.4pt}
\renewcommand{\footrulewidth}{0pt}

% ========================== Package Einstellungen & Sonstiges ========================== 

%Besondere Trennungen

%römische Aufzählungen mit \RM{Zahl}
\newcommand{\RM}[1]{\MakeUppercase{\romannumeral #1}}
\newcommand{\MATLAB}{\textsc{Matlab}\xspace}

\newcommand{\seeref}[1]{\seename \ \cref{#1}}

\renewcommand{\theequation}{\thesection.\arabic{equation}}





%Glossar entries
\makeglossaries
\renewcommand*{\glstextformat}{\textbf}

\newglossaryentry{turklingelanlage}{name=Türklingelanlage, description={Gesamtheit der Komponenten die denn Zusammen den Endprodukt darstellen}}
\newglossaryentry{ip}{name=IP, description={Internet Protocol}}
\newglossaryentry{dsl}{name=DSL, description={Digital Subscriber Line}}
\newglossaryentry{lte}{name=LTE, description={Long Term Evolution}}
\newglossaryentry{hd}{name=HD, description={High Definition}}
\newglossaryentry{sip}{name=SIP, description={Session Initiation Protocol}}
\newglossaryentry{webrtc}{name=WebRTC, description={Web Real-Time Communication}}
\newglossaryentry{aussensprechstelle}{name=Aussensprechstelle, description={Mikrocontroller mit verschiedeneModulen die an den Eingangstüre installiert wird}}
\newglossaryentry{poe}{name=PoE, description={Power over Ethernet}}
\newglossaryentry{hw}{name=HW, description={Hardware}}
\newglossaryentry{gpio}{name=GPIO, description={General purpose input/output}}
\newglossaryentry{usb}{name=USB, description={Universal Serial Bus}}
\newglossaryentry{php}{name=PHP, description={Hypertext Preprocessor}}
\newglossaryentry{html}{name=HTML, description={Hypertext Markup Language}}
\newglossaryentry{css}{name=CSS, description={Cascading Style Sheets}}
\newglossaryentry{tls}{name=TLS, description={Transport Layer Security}}
\newglossaryentry{gui}{name=GUI, description={Graphical user interface}}
\newglossaryentry{clientapp}{name=Client Applikation, description={Die Web Applikation welches auf dem Smartphone oder Tablet des Bewohner ausgeführt wird}}
\newglossaryentry{https}{name=HTTPS, description={Hypertext Transfer Protocol Secure}}
\newglossaryentry{voip}{name=VoIP, description={Voice over IP}}
\newglossaryentry{p2p}{name=P2P, description={Peer to Peer}}
\newglossaryentry{api}{name=API, description={Application programming interface}}
\newglossaryentry{ice}{name=ICE, description={Interactive Connectivity Establishment}}
\newglossaryentry{nat}{name=NAT, description={Network address translation}}
\newglossaryentry{stun}{name=STUN, description={Session Traversal Utilities}}
\newglossaryentry{v8}{name=V8, description={Standard zur Videokompression}}
\newglossaryentry{h264}{name=H.264, description={Standard zur Videokompression}}
\newglossaryentry{os}{name=OS, description={Operating system}}
\newglossaryentry{managementtool}{name=Management Tool, description={Webapplikation zur Erfassung von \gls{aussensprechstelle} und Bewohner}}
\newglossaryentry{mysql}{name=MySQL, description={eines der weltweit verbreitetsten relationalen Datenbankverwaltungssysteme}}
\newglossaryentry{arm}{name=ARM, description={Advanced RISC Machines}}
\newglossaryentry{lxde}{name=LXDE, description={freie Desktop-Umgebung für Unix}}
\newglossaryentry{json}{name=JSON, description={JavaScript Object Notation}}
%\gls{ip}

%lit
%\usepackage[style=alphabetic]{biblatex} 
\usepackage[backend=biber]{biblatex}
\addbibresource{database.bib}

\begin{document}


\pagestyle{empty}
\begin{titlepage}
%\ThisTileWallPaper{\paperwidth}{\paperheight}{image/Titleimage.png}


	\begin{flushleft}
		\includegraphics[scale=0.1]{image/ntb.jpg}
	\end{flushleft}
	
    \begin{center}
	    \vspace{0.5cm}
	    \Huge \textbf{\textsf{Türklingelanlage mit Standardkomponenten}} \\
		\vspace{0.5cm}
		\large\textbf{\textsc{Federico Crameri, Geo Bontognali}}\\
		
		\vspace{0.5cm}
	    \includegraphics[height=7cm]{image/Titleimage.png}
	
	    \normalsize
	    \vspace{1cm}
	    \large \textbf{Bachelorarbeit 2017}\\
	    \vspace{1cm}
	
	 \normalsize
	 	{
			\begin{tabular}{lll}
				\textbf{Studiengang:} & & Systemtechnik\\
				\textbf{Profil:} & & Informations- und Kommunikationssysteme\\
				\textbf{Referent:} & & Prof. Dr. Hauser-Ehninger Ulrich, MSc in Electronic Engineering\\
				\textbf{ } & & ulrich.hauser@htwchur.ch, +41 (0)81 286 39 97\\
				\textbf{Korreferent:} & & Toggenburger Lukas, Master of Science FHO in Engineering\\
				\textbf{ } & & lukas.toggenburger@htwchur.ch, +41 (0)81 286 27 22\\
				\textbf{Industriepartner:} & & SRM-Projects, Joeri Gredig\\
				\textbf{ } & & joeri.gredig@srm-projects.ch, +41 (0)79 340 44 85\\
			\end{tabular}
	    }
    \end{center}
\end{titlepage}

\cleardoublepage
% \part im Inhaltsverzeichnis nicht nummerieren
\makeatletter
\let\partbackup\l@part
\renewcommand*\l@part[2]{\partbackup{#1}{}}

% leere Seite einfügen nach dem Titel
\thispagestyle{empty}
\quad 
\newpage

\pagenumbering{Roman}
\pagestyle{plain}
\section*{Kurzfassung}
\label{sec:zusammenfassung}
%\addcontentsline{toc}{section}{Zusammenfassung}

In der heutigen Gesellschaft entwickelt sich alles mit erstaunlicher Geschwindigkeit. Diese rasante Entwicklung macht auch im Bereich der Gebäudetechnik keinen Halt. Die \gls{turklingelanlage}n, als Teil der Gebäudetechnik, sind auch davon betroffen, und haben sich in den letzten Jahren technologisch aber auch preislich weiterentwickelt. 
\\
\\
Im Rahmen dieser Bachelorarbeit haben wir einen Türklingelanlage-Prototyp mit Standardkomponenten entwickelt. Wichtig für unsere Arbeit war zu überprüfen, inwiefern die heutigen Standard- und Open Source Komponenten für ein solches System geeignet sind. Während der Entwurfsphase wurden unterschiedliche Anforderungen definiert. Die Kommunikation über die Türklingelanlage muss durch Video und Audio Signale erfolgen, die Gegensprechanlage wurde komplett auf der digitalen Ebene realisiert. In einer modernen Welt, wo jeder Mensch ständig mit dem Internet verbunden ist, ist eine digitale Lösung wohl der einzige richtige Weg. Dieser Weg hat unterschiedliche Herausforderungen mit sich gebracht. 
\\
\\
Das Endresultat ist eine digitale, flexible und zeitgemässe Türklingelanlage, welche mehrere Eingangstüren steuern und mit herkömmlichen Handys bedient werden kann
\section*{Abstract}
\label{sec:abstract}
%\addcontentsline{toc}{section}{Abstract}

This bachelor’s thesis is about to develop a tool, which can analyze, respectively optimize propeller-engine-systems from light to ultralight aircrafts and from paramotors. The goal was to create a tool, which provides further findings about propellers and their application area. Therefore, the given propeller geometry information from a CAD program is loaded and analyzed in the tool. There can also be made optionally geometry changes, substantially changes made to the twist of the propeller. Subsequently, the rotor system can be designed and optimized by calculation and graphical representation of the results for different flight situations.\\


There was also carried out verification with measurements of Helix Carbon GmbH, which gave evidence on the accuracy of the calculation. Some examples for optimization and analysis were explained in order to highlight possible applications.




\newpage

% leere Seite einfügen
\thispagestyle{empty}
\quad 
\newpage
%%Inhaltsverzeichnis
\tableofcontents
\newpage

%%Seitennummerierung neu beginnen, Zahlen [arabic], röm.Zahlen [roman,Roman], Buchstaben [alph,Alph]
\pagenumbering{arabic}
\newpage
\pagestyle{fancy}

\definecolor{snippetcolor}{gray}{0.9}






%\input{intro}
%\section{Chapter Example}
\label{sec:chapterexample}


Um Berechnungen rund um das Thema Propeller durchführen zu können, ist es nötig, die sogenannte Blade- und Momentum-Theorie zu verstehen und diese anwenden zu können. Die Ausführungen in dieser Arbeit orientieren sich stark an der Master-Thesis von Mario Heene\cite{heene}. Ebenfalls werden die wichtigsten Begriffe der Aerodynamik und Propellertheorie erläutert. 
%\subsection{Underchapter Example}
\label{subsec:underchapterexample}

Die Gestalt von Propellern kann durch viele 2D-Profile beschrieben werden. 2D-Profile gleichen dem Querschnitt eines Flügels. Durch den geringeren Druck auf der Oberseite des Profils resultieren Kräfte, die das Profil hoch drücken (siehe Kapitel \ref{subsec:momentumtheorie}, Einschub Bernoulli) oder im Falle des Propellers, Schub erzeugen. Für jedes Profil können die Kräfte dargestellt werden.

\vspace{0.05cm}

\begin{figure}[htb!]
\begin{center}
\includegraphics[width=0.48\textwidth]{airfoilKraefte}
\caption[Wirksame aerodynamische Kräfte an einem Profil]{Wirksame aerodynamische Kräfte an einem Profil \cite{aeroKraefte}. Die Auftriebskraft $L$ wirkt beim Propeller als Schub und die Widerstandskraft $D$ als Drehmoment.}
\label{fig:airfoilKraefte}
\end{center}
\end{figure}

\vspace{0.05cm}

In Abbildung \ref{fig:airfoilKraefte} sind die verschiedenen Kräfte dargestellt. $V_\infty$ stellt die Geschwindigkeit der Luft weit weg vom Körper dar. Die Kraft $R$ setzt sich zusammen aus der Auftriebskraft $L$ ('lift') und der Widerstandskraft $D$ ('drag'), wobei $L$ senkrecht zu $V_\infty$ steht und $D$ in die gleiche Richtung zeigt wie $V_\infty$.
Die Sehnenlänge $c$ ('chord') definiert die Länge des Profils von der Profilspitze bis zur Hinterkante. Dabei steht die Kraft $N$ senkrecht zu $c$ und $A$ in Richtung $c$. Der Winkel $\alpha$ ist der Winkel zwischen $V_\infty$ und $c$.
Über trigonometrische Funktionen können die Kräfte $N$ und $A$ folgendermassen beschrieben werden:

 

\section{Einführung}
\label{sec:chapterexample}

\subsection{Problemstellung}
\label{sec:chapterexample}

Heutzutage liefern diverse Hersteller verschiedene Lösungen für das Türglockensystem. Diese sind meistens Komplettsysteme, die nicht nur das einfache Klingel ermöglichen, sondern auch Zusatzfunktionen wie das Videostreaming anbieten. Diese Systeme sind aber meistens proprietär und werden für sehr hohe Preise verkauft.
\\
Die Komponenten, die für solche Systeme notwendig sind, sind aber heutzutage kostengünstig auf dem Markt erhältlich. Das Erarbeiten preiswerter Lösungen müsste somit möglich sein. 
\\
Natürlich spielen die Kosten einer Türsprechanlage auf die Investitionen eines Neubaus keine grosse Rolle. Sicher besteht aber in diesem Bereich eine Marktlücke und somit die Möglichkeit neue, bessere und günstigere Lösungen zu entwickeln.  

\subsection{Grundidee}
\label{sec:grundidee}

Die Grundidee dieser Arbeit ist es, durch das Zusammenspiel verschiedener Systemen/Technologien, eine kostengünstige und funktionale Türsprechanlage zu entwickeln.
\\ 
Um den Kostenfaktor zu berücksichtigen, soll die Anlage auf schon vorhandene Technologie/Hardware basieren. Somit fallen die hohen Kosten für die Beschaffung proprietärer Hardware weg.
\\
In der Zeit, in der die Hausautomation und das «Internet of things» immer mehr Bedeutung gewinnen, soll die Türsprechanlage diese Standards in Betracht ziehen.   
Dieses System soll den Benutzern ermöglichen, Ihre Sprechtüranlage durch herkömmliche Smartphone oder Tablet zu bedienen.
\\
Klingelt ein Besucher an der Eingangstüre, soll der Wohnungsbesitzer über sein Smartphone darauf aufmerksam gemacht werden. Über eine am Eingang installierte Kamera, bekommt er auch die Möglichkeit den Besucher im Streaming zu sehen und die Türe, falls erwünscht, durch einen Handybefehl zu öffnen.

\subsection{Lösungskonzept}
\label{sec:lösungskonzept}
Folgende Diagramm (?? referenz) zeigt eine Grobe Darstellung der verschiedenen Komponente die für den Türsprechanlage notwendig sind. Bei der Vorführung werden die Türöffner und die Glocken durch LEDs simulieret. 
An diese Stellen werden zwei Begriffe erklärt die in diesem Dokument von grosse Bedeutung sind. Die erte ist die Türsprechanlage, damit gemeint ist die Gesamtheit der Komponenten die denn Zusammen den Endprodukt darstellen.
Die zweite Begriff ist die Aussensprechstelle. Wie im Abbild ?? ersichtilich ist in Grunde genommen ein Microcontroller mit verschiedene Modulen die an den Eingangstüre installiert wird. Räumlich von der Aussensprechstelle getrennt befindet sich der Serverbereich. Diese besteht aus ein Microcontroller die als Server im Einsatz ist, ein Switch die dazu dient die Aussenssprechstelle mit Strom und Datenverbindung zu versorgen und die Relais die den Türöffner und Glocken betätigen.

\begin{figure}[htb!]
	\begin{center}
		\includegraphics[width=0.89\textwidth]{loesungskonzept}
		\caption[Lösungskonzept]{Lösungskonzept des Türsprechanlage}
		\label{fig:Lösungskonzept}
	\end{center}
\end{figure}
 
\newpage

\section{Projektplanung}
\label{sec:chapterexample}

\subsection{Prozess}
\label{sec:chapterexample}
Als Entwicklungsprozess wird ein hybrides Vorgehensmodell eingesetzt, welcher in Abbildung \ref{fig:hybridesModell} dargestellt wird. Im Rahmen einer Bachelorarbeit, in der die Anforderungen und Analysen schon im voraus im Fachmodul definiert worden sind, eignet sich am bestens ein lineares V-Modell. Ein solcher Prozess ist sehr schlank, übersichtlich und für diese Projektgrösse geeignet.
\\
Was das V-Modell nicht erlaubt, ist eine ständige Iteration mit dem Kunden während der Entwurf/Implementierungsphase. Daraus ergibt sich, wie im Abbild unten gezeigt, ein hybrides Modell welches uns zulässt, trotz der klar definierten Anforderungen, während der Entwurf- und der Implementierungsphase ein agiles Vorgehen mit dem Kunden durchzuführen.
\\
Die im Fachmodul geleistete Arbeit gehört zu den ersten zwei Phasen des Modells. Wie im linearen Vorgehensmodell vorgegeben, beginnt die nächste Phase der Arbeit sobald die vorherige Phase abgeschlossen ist. Die ganze Bachelorarbeit basiert auf Evaluationen und Entscheidungen, die in den ersten Phasen des Projekts getroffen worden sind. 

\begin{figure}[htb!]
	\begin{center}
		\includegraphics[width=0.89\textwidth]{hybridesModell}
		\caption[Hybrides Vorgehensmodell]{Hybrides Vorgehensmodell}
		\label{fig:hybridesModell}
	\end{center}
\end{figure}


\subsection{Zeitplanung}
\label{sec:zeitplanung}
Die folgenden Abbildungen stellen die Projektplanung und die Meilensteine zeitlich dar (\seeref{fig:projektPlanungAchse} \& \cref{fig:projektPlanung}). In die erste Woche werden die Hardwarekomponenten, die mittlerweile schon bestellt wurden, getestet und zusammengebaut. 
\\
Die nächste zwei Hauptpunkte betreffen die Programmierung der  Software, die in zwei Teile geteilt wurde.
\\ 
Beim Teil 1 geht es um die Skripts die serverseitig kleine Aufgaben übernehmen, beim Teil 2 geht es um die Programmierung der Software. Da werden die Webapplikationen entwickelt, die auf den Aussensprechstellen und auf den mobilen Geräten der Bewohner ausgeführt werden sollen.
\\
Die letzte Phase ist für die Optimierung und als Reserve gedacht.

\begin{figure}[htb!]
	\begin{center}
		\includegraphics[width=1\textwidth]{projektPlanAchse}
		\caption[Projektplanung Meilensteine]{Zeitplanung mit Meilensteine}
		\label{fig:projektPlanungAchse}
	\end{center}
\end{figure}


\begin{figure}[htb!]
	\begin{center}
		\includegraphics[width=0.9\textwidth]{projektPlanung}
		\caption[Projektplanung]{Projektplanung}
		\label{fig:projektPlanung}
	\end{center}
\end{figure}


\newpage

\section{Aktueller Stand der Technik}
\label{sec:chapterexample}

Eine Türsprechanlage, welche Audio und Video überträgt ist keine neue Erfindung. Auf dem Markt existieren bereits verschiedene Lösungen und das schon seit mehreren Jahren. Diese sind aber meistens Analoge Systeme und verfügen über die Vorteile der Digitalisierung nicht. Die Steuerung über eine Mobileapplikation ist bei solche Lösungen aus diesem Grund ausgeschlossen.

\begin{figure}[htb!]
	\begin{center}
		\includegraphics[width=0.66\textwidth]{analog_intercom}
		\caption[Analoge Türsprechanlage mit In-House Display]{Analoge Türsprechanlage mit In-House Display}
		\label{fig:analoge_intercom}
	\end{center}
\end{figure}

In den letzten Jahren sind die ersten Digitale Lösungen mit IP Videoübertragung auf dem Markt gekommen. Die Digitalisierung in diesem Bereich hat es die gigantische Schritten im Bereich der Miniaturisierung und die immer schnellere Internet Zugänge (xDSL, LTE, usw) zu verdanken.

\subsection{Die Herausforderungen der Digitalisierung}
Die Digitalisierung bringt nicht nur Vorteile mit sich. Besonders bei der Video und Audioübertragung. Während eine Analoge Videoübertragung ziemlich mühelos erfolgt muss im Fall eine Digitale Lösung das Video zuerst kodiert und dann dekodiert werden.
\\
Die heutige Kodierung-Algorithmen ermöglichen eine ziemlich schnelle Dekodierung. Mittlerweile hat jeder Smartphone genug Leistung um ein Full-HD Videostream vom Youtube oder Netflix in real-time zu dekodieren. Auf die andere Seite ist die Kodierung ein sehr rechenintensiven Prozess und benötigt sehr viel Leistung.
\\
Jeder der schon mal mit Video-Editing zu tun hatte, weisst, wie viel Zeit die Exportierung eines Video dauern kann.
\\
Die grösste Herausforderung für die real-time Digitale Video/Audio Kommunikation besteht also darin, die Kodierung und Dekodierung der Audio und Video Signal im vernünftigen Zeit durchzuführen. 

\subsection{Marktsituation}
\label{sec:chapterexample}
Der Hauptziel dieses Bachelorarbeit ist, eine Kostengünstige Lösung für eine digitale, flexible und skalierbare Gegensprechanlage. Tatsächlich ist es so, dass die bestehende Lösungen sehr teuer sind. Viele Produkte basieren auf Drittanbieter, SIP Gateways oder andere Elemente die Zusatzkosten verursachen. Das möchten wir alles vermeiden.

\begin{figure}[htb!]
	\begin{center}
		\includegraphics[width=0.33\textwidth]{myintercom}
		\caption[Telecom Behnkle MyIntercom]{Telecom Behnkle MyIntercom}
		\label{fig:myintercom}
	\end{center}
\end{figure}

Eine der günstigsten Produkte den wir finden konnten ist das \textit{"MyIntercom"} von Telecom Behnkle (\seeref{fig:myintercom}). Diese Türklingelanlage ist ziemlich flexibel und bietet die Möglichkeit, mehrere Türen anzuschliessen. Der Preis liegt hier bei zirka 1'600.- CHF pro Türe bei dem Basic-Modell.
\\
\\
 Dank der Aufschwung von Open-Source Hardware wie das Raspberry PI und Real Time Communication Protokolle wie WebRTC muss es möglich sein, kostengünstigere Lösungen zu erarbeiten. Bei den folgenden Kapiteln geht es nun um die effektive Realisierung einem Prototyp, welches die oben genannte Problemen adressiert. 

\newpage
\section{Hardware}
\label{sec:chapterexample}

Das System wird Hardwareseitig in zwei Teile unterteilt. Der Server, die zentrale Einheit und die Aussensprechstelle. An beide Orte wird eine Raspberry Pi 3 und das nötige Hardware eingesetzt.

\subsection{Server}
\label{sec:chapterexample}

Der Server wird mit einem Relay-Board verbunden. Diese wird die Gongs und die Türöffner bedienen. An dieser Stelle ist die Hardware-Konfiguration sehr einfach. Je nach wie viele Gongs und Türe verbunden werden müssen, könnten bis zwei 8-Channel Relayboards verbunden werden.
\\

<-- ABBILDUNG CON LA PI E I RELAY E I PIN A CUI SONO COLLEGATI
\\
\\

\subsection{Aussensprechstelle}
\label{sec:chapterexample}

Bei der Aussensprechstelle wird auch eine Raspberry Pi eingesetzt. Hier sind mehrere Zusatzkomponenten notwendig. Die Speisung an dieser stelle erfolgt nur über PoE, aus diesem Grund ist PoE-Splitter vorhanden.
\\
\\
Für die Audiowiedergabe ist ein kleines Lautsprecher und ein Verstärker nötig. Die Chinch-Anschluss der Raspberry Pi hat eine zu kleine innere Widerstand um direkt ein solches Lautsprecher anschliessen zu können. Die Hauptproblematik nun besteht darin, dass die Massen des Raspberry Pi, der Verstärker und des Audio-Interface alle zusammen gekoppelt sind. Das führt zu Brunschleifen die wiederum Störsignale auf dem Audio-Ausgang erzeugen. Um das zu vermeiden ist eine Massentrennfilter an dieser Stelle notwendig.
\\

<-- ABBILDUNG CON LA PI E COMPONENTI COLLEGATI
\\
\\

\subsection{PoE}
\label{sec:poe}
Moderne Hausalte werden meistens mit ethernet Verkabelung verlegt. Ziel des Aussensprechstelle ist die Installationskosten zu senken und die Montage zu vereinfachen. Drei Anschlusse werden von den Aussensprechstelle benötigt um sein Ziel zu erreichen und zwar Strom, Internetverbindung und eine Leitung der für den Türöffner zuständig ist. Alle diese Fünktionalität können in einem Kat 7 Ethernet Kabel zusammengeführt werden. 
\\
Cisco Catalyst 3560g welcher für den PoE Stromversorgung zuständigt ist verwendet das Phantomspeisung oder Mode A. Das heisst dass die mit Datenübertragung belegten Adern mit der Stromversorgung überlagert werden. Diese ist möglich da Elektrizität hat eine niedrige Frequenz von 60 Hz und Datenübertragungen im bereich 10-100MHz liegt.

\begin{figure}[htb!]
	\begin{center}
		\includegraphics[width=0.89\textwidth]{CatalystPoEpinouts}
		\caption[Catalyst 3560g PoE Pinbelegung]{Catalyst Pinouts}
		\label{fig:catalystPinouts}
	\end{center}
\end{figure}


Die einzige Nachteil bei dieser Konfiguration 

<-- ABBILDUNG CON IL CAVO ETHERNET
\\


\newpage
\section{Software}
\label{sec:chapterexample}

\subsection{Programmiersprachen}
Das System besteht aus mehrere Programme und Dienste. Für die Entwicklung werden folgende Programmiersprachen eingesetzt:
\begin{itemize}
	\item Java
	\item Javascript
	\item PHP
\end{itemize}
Im Verbindung mit PHP kommt natürlich die Markup-Languages HTML5/CSS, welche für die graphische Darstellung der Webapplikationen notwendig ist.

\subsubsection{Java}
\label{kap:java}
Alle Dienste die Serverseitig und ohne Interaktion mit dem Enduser ausgeführt werden, werden in Java programmiert. Als stark typisierte und Objektorientierte Programmiersprache eignet sich Java für dieses Projekt. Für Java sind auch unzählige Libraries verfügbar, insbesondere für die Hardware Steuerung der Raspberry Pi. Eine zweite Variante wäre Python gewesen, die auch das Raspberry sehr gut unterstüzt. Python ist aber zu wenig typisiert und für eher kleinere Softwarestücke gedacht.

\subsubsection{PHP/Javascript}
Die Enduser Applikation sowohl auch die Applikation bei der Aussensprechstelle werden Web-Applikationen sein. Dies ermöglicht eine schnelle und zeitgemässe Softwareentwicklung. Für dieses Projekt ist die System-Eingriffstiefe von Webapplikationen jedenfalls ausreichend. Es muss lediglich Zugriff auf Mikrofon, Lautsprecher und Kamera garantiert werden. Ein weiteres Punkt zugunsten einer Webapplikation ist die Cross-Plattform Kompatibilität. 
\\
Aus diesem Grund haben wir uns für PHP (Objektorientiert) im Kombination mit Javascript/HTML/CSS Entschieden. Eine zweite Variante wäre Java EE gewesen. Java EE eignet sich aber vor allem für grosse Softwarelösungen und bietet als gesamten Framework vieles mehr als was dieses Projekt benötigt. 
\\
\subsubsection{PHP Framework: Laravel}
Für die Entwicklung der Webapplikationen wird Laravel als PHP Framework eingesetzt. Laravel ist ein Open-Source PHP Web-Application-Framework, die sich für kleine bis zu mittelgrosse Projekte eignet. Laravel beruht auf dem Modell-View-Controller-Muster und ermöglicht eine Objektorientierte Programmierung in PHP.

\subsection{Software-Ecosystem}
Das System besteht aus mehrere Hardware- und Softwarekomponenten die zusammenarbeiten müssen (\seeref{fig:echosystem}). Die Kommunikation zwischen den Knoten ist von TLS immer gewährleistet. Die einzelne Komponente werden in den nächsten Kapiteln genauer beschrieben.

\begin{figure}[htb!]
	\begin{center}
		\includegraphics[width=1\textwidth]{ecosystem}
		\caption[Software / Hardware Ecosystem]{Software / Hardware Ecosystem}
		\label{fig:echosystem}
	\end{center}
\end{figure}

Die Software wird in zwei Gruppen unterteilt. Einerseits gibt es alle Dienste/Services \textit{(Violett)} die Lokal ausgeführt werden und quasi das Backend des Systems darstellen (\seeref{kap:dienste}). 
\\
Die zweite Gruppe beinhaltet die Webapplikationen \textit{(Grün)}, die eine GUI besitzen und für die Interaktion mit dem System gedacht sind. Darunter zählen die Client-App für den Bewohner, die Applikation bei der Aussensprechstelle wo die Bewohner angezeigt werden und das Management Tool. (\seeref{kap:webapp})
\\
Die Audio-Kommunikation zwischen die Aussensprechstellen und die Client-Apps wird mithilfe von WebRTC realisiert. Diese hat eine gewisse Komplexität und wird in ein eigenes Kapitel (\seeref{kap:webrtc}) behandelt.

\subsection{Dienste}
\label{kap:dienste}


\subsubsection{Keymapper}
Die Aussensprechstelle wird durch 3 Schalter bedient. Die Aufgabe des Keymappers besteht darin, bei einem Tastendruck eine Aktion auf der Aussensprechselle ausgeführt wird. Die drei Schalter wurden an GPIO Pins des Raspberry PI angeschlossen. Wie im \cref{kap:java}
beschrieben wurde, wird Java eingesetzt. 
\\
Die ursprüngliche Idee war der Keymapper als Daemon im Hintergrund laufen zu lassen. Diese würde der Vorteil haben das der Daemon mittels run, stop und restart stabil gesteuert werden könnte. Da die eingesetzte java Library zum laufen zwingend ein X server benötigt, wurde der Keymapper im Init level 5 wo auch der X Server ausgeführt. Die Ausführung der Software ergibt in diesem Fall eine Fehlermeldung da die X Server noch nicht vollständig initialisiert ist. 
Das Problem lag darin das per Definition ein Daemon Benutzerunabhängig ist. Somit steht der Library die auf dem X Server zugreift die wiederum Benutzerspezifisch ist, in Konflikt mit der Definition. 
\\
Die Desktop-Umgebung LXDE welche von Raspbian verwendet wird, bietet ein Autostart welche die Keymapper nach dem Initialisierung des X Server ausführt.

\subsubsection{Speaker Controller}
...
\subsubsection{Relay Controller}
...
\subsubsection{Watchdog}
...

\subsubsection{Signaling Server}
Der Signaling-Server ist ein bestandteil von WebRTC und wird in ein eigenes Kapitel ausführlich beschrieben (\seeref{kap:webrtc}).


\subsection{Webapplikationen}
\label{kap:webapp}


\subsubsection{Client Webapplikation}
Der Bewohner muss über eine Applikation verfügen, die auf dem Tablet oder Handy ausführbar sein muss. Mithilfe dieser App muss der Enduser folgendes können: Sich mit alle Aussensprechstellen verbinden können, ein Video Signal von der Kamera aller Eingänge erhalten, alle Türe öffnen und mit der Person bei der Türe über die Anlage kommunizieren können.
\\
\begin{figure}[htb!]
	\begin{center}
		\includegraphics[width=0.35\textwidth]{clientDemo}
		\caption[Design der Client-Webapp]{Design der Client-Webapp}
		\label{fig:clientDemo}
	\end{center}
\end{figure}
\\
Die \cref{fig:clientDemo} zeigt das Design für diese Webapplikation. Hier gezeigt ist die Smartphone Version. Dank ein Responsive-Design wird die selbe Applikation auch auf andere Geräte wie z.B. Tablets oder Computers passend angezeigt.
\\ 
Bei der Design-Entwurf standen Übersichtlichkeit und Benutzerfreundlichkeit im Vordergrund. Aus diesem Grund werden die Tasten für die Audio-Kommunikation und für die Öffnung der Türe gross Angezeigt. Das Videostream von der ausgewählte Türe wird sofort angezeigt und benötigt keine weitere Interaktion. 


\subsection{WebRTC}
\label{kap:webrtc}

\newpage
\section{Testabnahme}
\label{sec:testabnahme}
Überprühfung der Anforderungen
\newpage
\section{Fazit und Ausblick}
\label{sec:fazit}
Der Fazit wir hier noch hinzugefühgt.
\newpage
\section{Anhang}
\label{sec:anhang}
Diese Anleitungen sind an den Weiterentwikler des Prototyp gerichtet. Mithilfe von diese Dokumentation und den mitgelieferte Image des Aussensprechstelle, soll ein Entwickler im Stand sein ein fresh installiert Raspbian \gls{os} zu einer Aussensprechstelle/Server zu konfigurieren. Alles was Konfiguriert wurde, wurde Dokumentiert und in den Anleitung aufgeführt. Diese soll auch das Hinzufügen von zukünftige Funktionalitäten erleichtern. 

\subsection{Aussensprechstelle Konfigurationsanleitung}

\subsubsection{Aktuelle Stand}
Betriebssystem:	Raspbian jessie with pixel\\
Version: April 2017\\
Kernel Version: 4.4

\subsubsection{Namen und Passwortkonzept}
Hostname: DoorPixxx (x= fortlaufende Nummerierung)\\
User: pi\\
Password: bachelor (Einfachheitshalber wurde diese schwach Passwort ausgewählt. Sollte aber bei eine Produktive inbetriebnahme zwingend geändert werden)\\

\subsubsection{Betriebssystem Installation}
\begin{itemize}
	\item Das Image von raspberry.com herunterladen und extrahieren. (https://www.raspberrypi.org/downloads/raspbian/)
	\item Um die Image auf der SD Karte zu bringen benutzt man Etcher. (https://etcher.io/)
	\item Mit den Standard-Anmeldedaten Anmelden. User: pi Password: raspberry
\end{itemize}



\subsubsection{Allgemeine Einstellungen}

Der System soll auf dem neuste Stand aktualisieren werden
\begin{lstlisting}[backgroundcolor = \color{snippetcolor},
language = bash,
xleftmargin = 1cm,
framexleftmargin = 0.1em,
breaklines=true]
	apt-get update 
	sudo apt-get upgrade
\end{lstlisting}

Mit den Terminal Kommando 'sudo raspi-config' können durch eine grafische Oberfläche folgende allgemeine Einstellungen angepasst werden:
\begin{itemize}
	\item Unter 'Interfacing Options' muss die SSH Server aktiviert werden.
	\item Hostname gemäss Namenskonzept anpassen
	\item Neue Passwort für den Pi Benutzer gemäss Passwordkonzept setzen.
	\item Zum schluss soll noch die Zeit-Zone, den Land und die Tastatuslayout angepasst werden.
\end{itemize}

\subsubsection{Bidschirm Konfiguration}
Die Display Treiber von waveshare.com herunterladen und auf dem SD Karte in Root Directory speichern. (http://www.waveshare.com/wiki/4inch\char`_HDMI\char`_LCD)\\
Mit folgenden bash Kommandos wird der Treiber Instelliert:
\begin{lstlisting}[backgroundcolor = \color{snippetcolor},
language = bash,
xleftmargin = 1cm,
framexleftmargin = 0.1em,
breaklines=true]
	tar xzvf /boot/LCD-show-YYMMDD.tar.gz 
	cd LCD-show/
	chmod +x LCD4-800x480-show
	./LCD4-800x480-show
\end{lstlisting}
Nachdem das der Bildschirm Treiber installiert wurde, müssen die Einstellungen für den Bildschirm angepasst werden. Folgende Code-Zeilen müssen am ende des 'config.txt' Datei der sich in den root directory befindet, hinzugefügt werden.
\begin{lstlisting}[backgroundcolor = \color{snippetcolor},
language = bash,
xleftmargin = 1cm,
framexleftmargin = 0.1em,
breaklines=true]
	hdmi_group=2
	hdmi_mode=87
	hdmi_cvt 480 800 60 6 0 0 0
	dtoverlay=ads7846,cs=1,penirq=25,penirq_pull=2,
	speed=50000,keep_vref_on=0,swapxy=0,pmax=255,
	xohms=150,xmin=200,xmax=3900,ymin=200,ymax=3900
	display_rotate=3
\end{lstlisting}

\subsubsection{Browser Kiosk-mode}
Als erstes wir die unclutter tool installiert um den Mausepfeil auszublenden.
\begin{lstlisting}[backgroundcolor = \color{snippetcolor},
language = bash,
xleftmargin = 1cm,
framexleftmargin = 0.1em,
breaklines=true]
	sudo apt-get install unclutter
\end{lstlisting}
Kios-mode Einstellungen werden im config Datei (/home/pi/.config/lxsession/LXDE-pi/autostart) wie folgendes angepasst.
\begin{lstlisting}[backgroundcolor = \color{snippetcolor},
language = bash,
xleftmargin = 1cm,
framexleftmargin = 0.1em,
breaklines=true]
	# Chromium auto start in kiosk mode
	# path: /home/pi/.config/lxsession/LXDE-pi/autostart
	@lxpanel --profile LXDE-pi
	@pcmanfm --desktop --profile LXDE-pi
	#@xscreensaver -no-splash
	@point-rpi
	@xset s off
	@xset s noblank
	@xset -dpms
	@chromium-browser --noerrdialogs --kiosk --incognito https://172.16.111.99/server
\end{lstlisting}

\subsubsection{Aussensprechstelle Initialisierung}
Im Homeverzeichnis unter .config/autostart wird die Datei Aussensprechstelle.desktop erstellt.
\begin{lstlisting}[backgroundcolor = \color{snippetcolor},
language = bash,
xleftmargin = 1cm,
framexleftmargin = 0.1em,
breaklines=true]
	touch /home/pi/Aussensprechstelle/Startup/AussensprechstelleLauncher.sh
\end{lstlisting}
Inhalt der Script:
\begin{lstlisting}[backgroundcolor = \color{snippetcolor},
language = bash,
xleftmargin = 1cm,
framexleftmargin = 0.1em,
breaklines=true]
	#!/bin/bash
	# This script executes the needed commands on startup to initialize the Aussensprechstelle
	# /home/pi/Aussensprechstelle/Startup/AussensprechstelleLauncher.sh
	#
	# Activates the Camera Driver (Safe mode because of the chrome resolution bug)
	sudo modprobe bcm2835-v4l2 gst_v4l2src_is_broken=1
	#
	# Clears the old TasterController PID of the process (In case of system shutdown)
	file="/var/run/TasterController.pid"
	if [ -f $file ] ; then
		rm $file
	fi
	#
	# Starts the TasterController
	sudo java -jar /home/pi/Aussensprechstelle/TasterController/TasterController.jar &
	#
	# Creates the PID for the taster controller
	sudo echo $! > /var/run/TasterController.pid
	#
	# Starts the watchdog service
	sudo service watchdog start	
\end{lstlisting}

\subsubsection{Taster Controller}
Die Tastencontroller die für den Key Mapping zuständig ist wird von dem oben gezeigte AussensprechstelleLauncher.sh unter /home/pi/Aussensprechstelle/TasterController/TasterController.jar gestartet. Also muss die kompilierte Jar Artefakt dorthin kopiert werden. \\
Folgende GPIO Pins werden von den 3 Tasten benötigt um die Aussensprechstelle zu steuern.
\begin{itemize}
	\item GPIO17(16) simuliert den Tastendruck J «Links navigieren»
	\item GPIO27(20) simuliert den Tastendruck K «Anrufen»
	\item GPIO22(21) simuliert den Tastendruck L «Rechts navigieren»
\end{itemize}

\subsubsection{Speaker Controller Service}
Als erstes muss der mitgelieferte Jar Artefakt SpeakerController.jar unter folgendes Pfad kopiert werden:
\begin{lstlisting}[backgroundcolor = \color{snippetcolor},
language = bash,
xleftmargin = 1cm,
framexleftmargin = 0.1em,
breaklines=true]
	/home/door/Aussensprechstelle/SpeakerController/SpeakerController.jar
\end{lstlisting}
Um den SpeakerController als Service unter Unix laufen zu lassen muss unter /etc/init.d/ der speakerController Script erzeugt werden. Der Inhalt des Script wird mit den Projekt mitgeliefert.
Um es ausführbar zu machen muss noch die «execute» Berechtigung gegeben werden
\begin{lstlisting}[backgroundcolor = \color{snippetcolor},
language = bash,
xleftmargin = 1cm,
framexleftmargin = 0.1em,
breaklines=true]
	touch /etc/init.d/speakerController
	chmod +x /etc/init.d/speakerController
\end{lstlisting}
Damit der speakerController Service auch automatisch beim Systemstart  ausgeführt wird muss noch folgendes Kommando ausgeführt werden:
\begin{lstlisting}[backgroundcolor = \color{snippetcolor},
language = bash,
xleftmargin = 1cm,
framexleftmargin = 0.1em,
breaklines=true]
	sudo update-rc.d speakerController defaults
\end{lstlisting}
Der Speaker Controller kann nun mit folgende commandos gestartet und gestoppt werden
\begin{lstlisting}[backgroundcolor = \color{snippetcolor},
language = bash,
xleftmargin = 1cm,
framexleftmargin = 0.1em,
breaklines=true]
	sudo service speakerController start
	sudo service speakerController stop
	sudo service speakerController restart
	sudo service speakerController status
\end{lstlisting}

\subsubsection{Watchdog/Watchdog deamon}
Um die von den Aussensprechstelle benötigte Dienste zu monitorieren die es benötigt wird ein Watchdog verwendet. Raspberry Pi hat ein «stad-alone» Hardware Watchdog die ein Autostart durchführt sobald eine der Dienste oder den OS steht. 
Mit folgende Kommandos wird der watchdog installiert:
\begin{lstlisting}[backgroundcolor = \color{snippetcolor},
language = bash,
xleftmargin = 1cm,
framexleftmargin = 0.1em,
breaklines=true]
	sudo modprobe bcm2835-wdt
	sudo apt-get install watchdog chkconfig
	sudo chkconfig watchdog on
	sudo /etc/init.d/watchdog start
\end{lstlisting}
Damit die SpeakerController und die TasterController von den Watchdog überwachen werden muss unter /etc/watchdog.conf die Konfiguratiosdatei abgeändert werden. Der Inhalt des Konfigurationsdatei wird mit dem Projekt mitgeliefert. 


\subsection{Server Konfigurationsanleitung}
\subsubsection{Aktuelle Stand}
Betriebssystem:	Raspbian jessie with pixel\\
Version: April 2017\\
Kernel Version: 4.4

\subsubsection{Namen und Passwortkonzept}
Hostname: SrvPixxx (x= fortlaufende Nummerierung)\\
User: pi\\
Password: raspberry (Default Password)\\


\subsubsection{Software Installation}
Nun werden die benötigten Dienste und Tools installiert, die von dem Server benötigt werden.

\begin{itemize}
	\item Installation der Webserver. (PHP, Nginx, MySQL, Java SDK, Composer, Utils)
\end{itemize}

\begin{lstlisting}[backgroundcolor = \color{snippetcolor},
language = bash,
xleftmargin = 0.5cm,
framexleftmargin = 0.1em,
breaklines=true]
	apt-get install nginx
	udo apt-get install mysql-server
	apt-get install php5-fpm php5-mysql
	sudo apt-get install mysql-server mysql-client
	sudo apt-get install oracle-java8-jdk
	sudo apt-get install curl php5-cli git
	curl -sS https://getcomposer.org/installer | sudo php -- --install-dir=/usr/local/bin --filename=composer
\end{lstlisting}

\subsubsection{Erstellung SSL Zertifikate}
Bevor die Webapplikationen installiert werden können, müssen die Zertifikate generiert werden. (Self-Signed) 

\begin{itemize}
	\item Erstellung SSL Zertifikat für die Client Webapplikation.
	Als Hostname wird hier als Beispiel intercom.app verwendet. Zuerst muss eine Konfigurationsdatei (v3.ext) mit dem folgenden Inhalt generiert werden:
\end{itemize}

\begin{lstlisting}[backgroundcolor = \color{snippetcolor},
language = bash,
xleftmargin = 0.5cm,
framexleftmargin = 0.1em,
breaklines=true]
	authorityKeyIdentifier=keyid,issuer
	basicConstraints=CA:TRUE
	keyUsage = digitalSignature, nonRepudiation, keyEncipherment, dataEncipherment
	subjectAltName = @alt_names
	
	[alt_names]
	DNS.1 = intercom.app
\end{lstlisting}
Falls ein IP Als hostname verwendet wird, kann man \textit{@alt\_names} mit IP:192.168.0.18 ersetzen.

\begin{itemize}
	\item Nun müssen folgende Befehle eingegeben werden. Wenn gefragt, muss der Hostname oder IP als Common Name (CN) Eingegeben werden. Als Password kann immer dieselbe verwendet werden und muss das Keystore Password der Signaling-Server entsprechen.
\end{itemize}

\begin{lstlisting}[backgroundcolor = \color{snippetcolor},
xleftmargin = 0.5cm,
framexleftmargin = 0.1em,
breaklines=true]
	sudo openssl genrsa -des3 -out rootCA.key 2048
	
	sudo openssl req -x509 -new -nodes -key rootCA.key -sha256 -days 1024 -out rootCA.pem
	
	sudo openssl req -new -sha256 -nodes -out server.csr -newkey rsa:2048 -keyout server.key
	
	sudo openssl x509 -req -in server.csr -CA rootCA.pem -CAkey rootCA.key -CAcreateserial -out server.crt -days 500 -sha256 -extfile v3.ext
	
	sudo openssl pkcs12 -export -in server.crt -inkey server.key -out cert.p12
	
	sudo keytool -importkeystore -srckeystore cert.p12 -srcstoretype PKCS12 -destkeystore keystore.jks -deststoretype JKS
	
	sudo openssl x509 -inform PEM -outform DER -in server.crt -out phone.der.crt

\end{lstlisting}

Somit wurden diverse Dateien generiert. Die folgenden werden später gebraucht.
\begin{itemize}
	\item server.cert und server.key -> SSL Zertifikate für Apache2
	\item rootCA.pem -> Root CA. Das muss in den Client-Browser importiert werden, damit die Clients den Server als Vertraulich erkennen. 
	\item phone.der.crt -> Root CA für Mobilegeräte. Bei Mobilegeräte kann es per E-Mail verschickt werden und dann aus der Systemeinstellungen installiert werden.
	\item keystore.jks -> Das muss später in die Selber Verzeichnis kopiert werden, wo der SignalingServer installiert wird.  
\end{itemize}

\subsubsection{Konfiguration von Nginx}
Vor dem Deploy den Webapplikationen muss der Webserver noch konfiguriert werden.
\\
\begin{itemize}
	\item In der Datei /etc/php5/fpm/php.ini muss die folgende Zeile auskommentiert und editiert werden: 
\end{itemize}

\begin{lstlisting}[backgroundcolor = \color{snippetcolor},
language = bash,
xleftmargin = 0.5cm,
framexleftmargin = 0.1em,
breaklines=true]
	cgi.fix_pathinfo=0
\end{lstlisting}

\begin{itemize}
	\item Nun muss Nginx so konfiguriert werden, dass PHP als compiler verwendet wird. Der Datei \textit{/etc/nginx/sites-available/default} muss editiert werden. Im rot markiert sind die Stellen die angepasst werden müssen.
	\item Nun müssen die zwei VirtualHosts für die zwei WebApps konfiguriert werden. Diese werden unter verschiedene Ports laufen. Für das müssen zwei Konfigurationsdatei unter \textit{/etc/nginx/sites-available} erstellt werden. Bsp: intercom.app und management.app.
	Der Inhalt muss wie folgendes aussehen. Im rot markiert sind die Stellen, die unterschiedlich sein müssen für die beiden Webapplikationen. Der Path zu den SSL-Zertifikate muss auch angepasst werden.	
\end{itemize}

\begin{lstlisting}[backgroundcolor = \color{snippetcolor},
language = bash,
xleftmargin = 0.5cm,
framexleftmargin = 0.1em,
breaklines=true]
# Default server configuration
server {
# SSL configuration
listen 443 ssl;  # 444 for the second host
listen [::]:443 ssl;

ssl_certificate /path/to/the/certificate/server.crt;
ssl_certificate_key /path/to/the/certificate/server.key;

root /var/www/intercom/public;  # /management/public for the second host

# Add index.php to the list if you are using PHP
index index.php index.html index.htm index.nginx-debian.html;

server_name _;

location / {
# First attempt to serve request as file, then
# as directory, then fall back to displaying a 404.
try_files $uri $uri/ /index.php?$query_string;
}

location ~ \.php$ {
include snippets/fastcgi-php.conf;
fastcgi_pass unix:/var/run/php5-fpm.sock;
}
location ~ /\.ht {
deny all;
}
}
\end{lstlisting}

\begin{itemize}
	\item Zum Abschliessen noch die folgenden Befehle eingeben:
\end{itemize}

\begin{lstlisting}[backgroundcolor = \color{snippetcolor},
language = bash,
xleftmargin = 0.5cm,
framexleftmargin = 0.1em,
breaklines=true]
sudo ln -s /etc/nginx/sites-available/intercom.app /etc/nginx/sites-enabled/
sud ln -s /etc/nginx/sites-available/management.app /etc/nginx/sites-enabled/
sudo service nginx reload
sudo service nginx restart
sudo service php5-fpm restart
sudo reboot

\end{lstlisting}

\subsubsection{Deploy Webapplikationen}
Vor dem Deploy den Webapplikationen muss der Webserver noch konfiguriert werden.
\\
\begin{itemize}
	\item Die beide Webapplikationen müssen zuerst auf dem Server unter den passenden Verzeichnissen kopiert werden.\\
		/var/www/management\\
		/var/www/intercom
	\item Rechte anpassen
\end{itemize}

\begin{lstlisting}[backgroundcolor = \color{snippetcolor},
language = bash,
xleftmargin = 0.5cm,
framexleftmargin = 0.1em,
breaklines=true]
sudo chmod -R 775 /var/www
sudo chmod -R 777 /var/www/management/storage
sudo chmod -R 775 /var/www/intercom/storage
sudo chgrp -R www-data /var/www/
\end{lstlisting}

\begin{itemize}
	\item MySQL Database erstellen, dann Anmeldedaten und Database Name in der Datei: .env eingeben. (Falls .env nicht vorhanden: cp .env.example .env)
	\item Webapplikation installieren: (Diese befehle müssen in der Root Dir von jede Webapp eingegeben werden).
\end{itemize}

\begin{lstlisting}[backgroundcolor = \color{snippetcolor},
language = bash,
xleftmargin = 0.5cm,
framexleftmargin = 0.1em,
breaklines=true]
composer install
php artisan migrate (MNGMT Tool Only)
php artisan key:generate
\end{lstlisting}
Die zwei Webapps müssten nun unter die ports 443 und 444 aufrufbar sein.

\subsubsection{Deploy Dienste und Services}
\begin{itemize}
	\item Zuerst die benötigten Pfade erstellen
\end{itemize}

\begin{lstlisting}[backgroundcolor = \color{snippetcolor},
language = bash,
xleftmargin = 0.5cm,
framexleftmargin = 0.1em,
breaklines=true]
mkdir /home/pi/server/signalingServer
mkdir /home/pi/server/relayController
\end{lstlisting}

\begin{itemize}
	\item Die beide kompilierte JARs in den entsprechenden Verzeichnissen kopieren. Die kompilierten JARs sind in den jeweilige Projekts Verzeichnisse unter /deploy zu finden.
	\item Nun muss noch für den SignalingServer noch das vorher hergestellte keystore.jks kopiert werden. Der Keystore muss sich in dem gleichen Verzeichnis wie der Signaling Server befinden.
	\item Für beide Dienste muss noch das autostart Skript unter /etc/init.d/ kopiert werden. Die Skripte sind unter dem Script-Verzeichnis gespeichert. Auch diese Skripte müssen ausführbar sein. Folgende Befehle müssen noch eingegeben werden:
\end{itemize}

\begin{lstlisting}[backgroundcolor = \color{snippetcolor},
language = bash,
xleftmargin = 0.5cm,
framexleftmargin = 0.1em,
breaklines=true]
sudo chmod +x /etc/init.d/signalingServer
sudo chmod +x /etc/init.d/relayController
sudo update-rc.d signalingServer defaults
sudo update-rc.d relayController defaults 

\end{lstlisting}

\subsection{Zertifikate}
....









%%Verzeichnis aller Bilder






%% Abbildungsverzeichnis, Tabellenverzeichnis, Abkürzungsverzeichnis
\listoffigures
\addcontentsline{toc}{section}{Abbildungsverzeichnis}
\newpage
\listoftables
\addcontentsline{toc}{section}{Tabellenverzeichnis}


\newpage

\thispagestyle{plain}
\section*{Abkürzungsverzeichnis}
\addcontentsline{toc}{section}{Abkürzungsverzeichnis}
\newpage
\thispagestyle{plain}

%\begin{center}
%\begin{minipage}[t]{1\textwidth}
\section*{Eidesstattliche Erklärung}
\addcontentsline{toc}{section}{Eidesstattliche Erklärung}
Die Verfasser dieser Bachelorarbeit, Federico Crameri und Geo Bontognali, bestätigen, dass sie die Arbeit selbstständig und nur unter Benützung der angeführten Quellen und Hilfsmittel angefertigt haben. Sämtliche Entlehnungen sind durch Quellenangaben festgehalten.
	
	
\vspace{2.5cm}

\hspace{2cm} Ort, Datum \hfill Geo Bontognali \hspace{2cm}

\vspace{4cm}

\hspace{2cm} Ort, Datum \hfill Federico Crameri \hspace{2cm}
%\end{minipage}
%\end{center}






% leere Seite einfügen
\thispagestyle{empty}
\quad 
\newpage

%%Literaturverzeichnis
\nocite{*}
\printbibliography
%\printbibliography 


\end{document}
% ======================================= Dokumentende =======================================