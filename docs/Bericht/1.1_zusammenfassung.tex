\section*{Zusammenfassung}
\label{sec:zusammenfassung}
%\addcontentsline{toc}{section}{Zusammenfassung}


In dieser Bachelorarbeit geht es darum ein Werkzeug/Tool zu entwickeln, das Propeller-Motor-Systeme von Leicht- bis Ultraleichtflugzeugen, sowie von Motorschirmen analysieren bzw. optimieren kann. Das Werkzeug soll weiterführende Erkenntnisse über Propeller und deren Einsatzbereich zur Verfügung stellen. Dazu werden Geometrieinformationen des Propellers von einem CAD-Programm ins Werkzeug geladen und ausgewertet. Gegebenenfalls können auch Geometrieänderungen, im Wesentlichen Änderungen an der Verdrillung des Propellers, vorgenommen werden. Anschliessend kann das Rotorsystem durch Berechnung und graphische Darstellung der Ergebnisse für verschiedene Flugsituationen ausgelegt bzw. optimiert werden. 
\\
Im Zentrum der Arbeit stand das Implementieren der Propellertheorie, welche einerseits die Blade Element Theorie und andererseits die Momentum Theorie beinhaltet, sowie das Einbinden eines Hilfsmittels, welches notwendige Koeffizienten zur Propellerberechnung liefert. Somit ermöglicht das Werkzeug dem Benutzer Schub, Drehmoment und weitere Grössen eines Propellers in verschiedenen Flugsituationen zu berechnen und damit Analysen durchzuführen. Konkret wurde das Werkzeug mit \MATLAB\cite{matlab} und Python\cite{python} umgesetzt.
\\
Ebenfalls wurde eine Verifikation mit Messdaten der Firma Helix Carbon GmbH durchgeführt, welche Hinweise auf die Richtigkeit der Berechnung gab. Auch wurden einige Beispiele zur Optimierung und zur Auslegung durchgeführt, um mögliche Anwendungsfälle aufzuzeigen.
