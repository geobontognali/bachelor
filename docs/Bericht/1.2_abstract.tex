\section*{Abstract}
\label{sec:abstract}
%\addcontentsline{toc}{section}{Abstract}

This bachelor’s thesis is about to develop a tool, which can analyze, respectively optimize propeller-engine-systems from light to ultralight aircrafts and from paramotors. The goal was to create a tool, which provides further findings about propellers and their application area. Therefore, the given propeller geometry information from a CAD program is loaded and analyzed in the tool. There can also be made optionally geometry changes, substantially changes made to the twist of the propeller. Subsequently, the rotor system can be designed and optimized by calculation and graphical representation of the results for different flight situations.\\

The main topic of the work was the implementation of the propeller’s theory, which is on the one hand the blade element theory and on the other hand the momentum theory as well as the inclusion of an auxiliary tool, which provides necessary coefficients for the propeller calculation. This allows the user to calculate and to perform analysis of thrust, torque and other values of a propeller in different flight situations. Specifically, the tool was implemented with \MATLAB\cite{matlab} and Python\cite{python}.\\

There was also carried out verification with measurements of Helix Carbon GmbH, which gave evidence on the accuracy of the calculation. Some examples for optimization and analysis were explained in order to highlight possible applications.



