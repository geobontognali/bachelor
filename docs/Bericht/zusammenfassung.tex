\section*{Kurzfassung}
\label{sec:zusammenfassung}
%\addcontentsline{toc}{section}{Zusammenfassung}

In der heutigen Gesellschaft entwickelt sich alles mit erstaunlicher Geschwindigkeit. Diese rasante Entwicklung macht auch im Bereich der Gebäudetechnik keinen Halt. Die \gls{turklingelanlage}n, als Teil der Gebäudetechnik, sind auch davon betroffen, und haben sich in den letzten Jahren technologisch aber auch preislich weiterentwickelt. 
\\
\\
Im Rahmen dieser Bachelorarbeit haben wir einen Türklingelanlage-Prototyp mit Standardkomponenten entwickelt. Wichtig für unsere Arbeit war zu überprüfen, inwiefern die heutigen Standard- und Open Source Komponenten für ein solches System geeignet sind. Während der Entwurfsphase wurden unterschiedliche Anforderungen definiert. Die Kommunikation über die Türklingelanlage muss durch Video und Audio Signale erfolgen, die Gegensprechanlage wurde komplett auf der digitalen Ebene realisiert. In einer modernen Welt, wo jeder Mensch ständig mit dem Internet verbunden ist, ist eine digitale Lösung wohl der einzige richtige Weg. Dieser Weg hat unterschiedliche Herausforderungen mit sich gebracht. 
\\
\\
Das Endresultat ist eine digitale, flexible und zeitgemässe Türklingelanlage, welche mehrere Eingangstüren steuern und mit herkömmlichen Handys bedient werden kann