\section*{Kurzfassung}
\label{sec:zusammenfassung}
%\addcontentsline{toc}{section}{Zusammenfassung}

In der heutigen Gesellschaft entwickelt sich alles mit erstaunlicher Geschwindigkeit. Die Gebäudetechnik, darunter zählen unterschiedliche Bereiche und Systeme, macht wohl keine Ausnahme. Und auch die \gls{turklingelanlage}n entwickeln sich mit.
\\
\\
Im Rahmen dieser Bachelorarbeit haben wir einen Türklingelanlage-Prototyp mit Standardkomponenten entwickelt. Wichtig für unsere Arbeit war zu überprüfen, inwiefern die heutigen Standard- und Open Source Komponenten für ein solches System geeignet sind. Während der Entwurfsphase wurden unterschiedliche Anforderungen definiert. Die Kommunikation über die Türklingelanlage muss durch Video und Audio Signale erfolgen, die Gegensprechanlage wurde komplett auf die digitale Ebene realisiert. In einer modernen Welt, wo jeder Mensch ständig mit dem Internet verbunden ist, ist eine digitale Lösung wohl der einzige richtige Weg. Das brachte unterschiedliche Herausforderungen mit sich.
\\
\\
Das Endresultat ist eine digitale, flexible und zeitgemässe Türklingelanlage, welche mehrere Eingangstüren steuern kann und mit herkömmlichen Handys bedient werden kann. 