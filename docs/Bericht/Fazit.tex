\section{Ausblick}
\label{sec:ausblick}
\subsection{Verbesserungs- und Erweiterungsmöglichkeiten}
\label{sec:erweiterungen}
Der aktuelle Stand des Systems bietet die Standardfunktionalitäten die notwendig sind um den Prototyp in einer Testumgebung einsetzen zu können. Skalierbarkeit und Weiterentwicklung waren deshalb zwei wichtige Bedingungen die die ganze Arbeit geprägt haben.
\\
Das Einsetzen von Webtechnologien hat dazu geführt, dass die verschiedenen Benutzerschnittstellen in Zukunft ohne grossen Aufwand und Know-How abgeändert oder erweitert werden können.
\\
In Bereich Hardware könnten ebenso Verbesserungen vorgenommen werden. Die Komponenten wurden so ausgewählt, dass eine schnelle und einfache Implementierung ermöglicht wird. Wie im Bericht schon erwähnt, sollte für die \gls{aussensprechstelle}n einen leistungsstärkeren Micro-Controller eingesetzt werden. Auch für die Kamera könnten weitere Produkte evaluiert werden, die für den Zweck besser geeinigt wären. Ein Kamerasensor mit integriertem Autofokus könnte zum Beispiel die Qualität des Bildes weiter verbessern.
\\
Die eingesetzten Komponenten wie der \gls{poe}-Splitter, der Massentrennfilter und der Verstärker sind Open Source Produkte. Mit den vorhandenen Schaltplänen dieser Komponenten, könnte man eine einzelne Platine anfertigen lassen, die alle Funktionalitäten beinhaltet. Diese würde die Montage deutlich vereinfachen und gleichzeitig auch das Ausfallsrisiko senken.
\\

\subsection{Einsatzmöglchkeiten}
\label{sec:eisatz}
Der entstandene Prototyp ist die richtige Lösung um die Machbarkeit einer solchen Digitalisierung zu demonstrieren. Die Anlage soll weiterentwickelt werden, und dieser Prototyp ist bestens geeignet um mögliche Interessenten und Investoren aufmerksam zu machen. Die kompakte Bauweise der \gls{turklingelanlage} und die einfachen Verkabelungstechnologien ermöglichen bei Fachmessen oder Präsentationen eine rasche Inbetriebnahme der Anlage.

\newpage

\section{Schlussfolgerung}
\label{sec:schlussfolgerung}
Das, während der Bachelorarbeit, entstandene Produkt hat gezeigt, dass die Digitalisierung einer \gls{turklingelanlage} mit der heutigen Technologie möglich ist. Was für einen Laien auf den ersten Blick als relativ einfaches System wahrgenommen wird, hat uns bei der Entwicklung grosse Herausforderung bereitgehalten. 
\\
\\
Der grösste Fehler aus unsere Seite war, teilweise viel zu \textit{Endprodukt-Orientiert} zu arbeiten. Beispielsweise haben wir sehr viel Zeit für das Design der Webapplikationen investiert. Für uns war es wichtig, eine Zeitgemässe und vorallem Benutzerfreundliche Benutzeroberfläche zu entwickeln. Im Nachhinein ist uns klar gewesen, dass wir uns mehr Zeit einplanen sollten, für die Kernfunktionalitäten. 
\\
\\
Im Gegensatz zur herkömmlichen, analogen Türklingelanlagen werden bei unserem Prototyp alle Video-Streams digitalisiert, codiert und decodiert. Diesen Schritt führt, im Gegensatz zu analogen Systemen welche in Echtzeit funktionieren, zu kleineren aber spürbaren Delays. So auch die Qualität der Videoübertragung. Beim aktuellen Stand der Technik ist eine bidirektionale in Echtzeit funktionierende HD-Verbindung kaum realisierbar. Die Qualität die erreicht wird, nährt sich denjenigen der Marktführer wie z.B Skype oder Facebook.
\\
In den letzten Jahren wurden enorme Schritte im Bereich der Technologien und der Forschung gemacht und der Trend ist stets positiv. Unserer Meinung nach wird aber in wenigen Jahren in der Welt der Hausautomatisierung die Kluft zwischen analoge und digitale Technologie überwunden werden.
\\
Im Verlauf der Arbeit wurde uns klar, dass der Weg zur Digitalisierung die richtige Entscheidung war. Die digitale Welt bring unzählige Vorteile mit sich, die in der zukünftigen Hausautomation nicht mehr wegzudenken sind. So ist zum Beispiel die Interaktion der \gls{turklingelanlage} über ein Smartphone oder ein Tablet ein erheblicher Vorteil im Gegensatz zu analoge Systeme. Weiter ermöglicht das digitale System der Zugriff auf die Videoübertragung von aussen, was mit analoge Technologie bis jetzt kaum realisierbar war.
\\
   

    



  
