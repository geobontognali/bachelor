\section{Ausblick}
\label{sec:ausblick}
\subsection{Verbesserungs- und Erweiterungsmöglichkeiten}
\label{sec:erweiterungen}
Der aktuelle Stand des System  bietet die Standardfunktionalitäten die notwendig sind um den Prototyp in einem Testumgebung einsetzen zu können. Skalabilität und Weiterentwicklung waren deshalb zwei wichtige Begriffe die den Ganzen Arbeit geprägt haben.\\
Das einsetzen von Webtechnologien hat dazu gebracht, das die verschiedene Benutzerschnittstellen im Zukunft, ohne grosse Aufwand und Konw-How abgeändert oder erweitert werden können. 
\\
In Bereich Hardware könnten ebenso verbesserungen vorgenommen werden. Die Komponenten wurden so ausgewählt um eine Schnelle und einfache Implementierung zu ermöglichen. Wie im Bericht schon erwähnt sollte für die \gls{aussensprechstelle}n eine Leistungstärkere Microcontroller eingesetzt werden. Auch für den Kamera könnten weitere Produkte ewaluiert werden die für den Zweck besser geeinigt sind. Ein Kamerasensor mit integrierte autofokus könnte zum Beispiel die Qualität des Bild witerhin verbessern.\\
Die eingesetzte Komponenten wie der \gls{poe} Splitter, der Massentrennfilter und den Verstärker sind Open Source. Mit den vorhandenen Schltplänen diese Komponenten, könnte man eine einzelne Platine anfertigen lassen, die alle Funktionalitäten beinhaltet. Diese würde den Montage deutlich  vereinfachen und gleichzeitig auch den Ausfallsrisiko sinken. 

\subsection{Einsatzmöglchkeiten}
\label{sec:eisatz}
Das entstandene Prototyp ist die richtige Lösung um das machbarkeit eine solche Digitalisierung zu demonstrieren. Die Anlage soll weiterentwickelt werden, und diese Prototyp ist bestens geeignet um mögliche Interessentetn und Investoren aufmerksam zu machen. Die kompakte bauweise des \gls{turklingelanlage} und die einfache Verkabelungstechnologien ermöglichen bei Fachmessen oder Presentationen eine rasche Inbetriebname des Anlage. 

\newpage

\section{Schlussfolgerung}
\label{sec:schlussfolgerung}
Das während dem Bachelorarbeit entstandene Produkt hat gezeigt das eine Digitalisierung von ein \gls{turklingelanlage} mit den heutige Technologie möglich ist. Was für ein Laie auf eine ersten blick, als relativ einfach System wahrgenommen wird, hat uns bei der Enwicklung grosse Herausforderung bereitgehalten. Im gegensatzt zum herkömmliche analoge Türklingelanlagen werden bei unserem Prototyp alle Viedeo Streams digitalisiert, codiert und decodiert. Diese Schritt führ im gegensatz zur analoge Systemen, welche im echzeit funktionieren, zu kleinere aber spürbare delays. So auch die Qualität des Videoübertragung. Bei den jetztige stand der Technik ist eine bidirektionale im echtzeit funktinierende HD Verbindung kaum realisierbar. Die Qualität die erricht wird nähr diejenigen von den Marktführer wie z.B Skype oder Facebook.\\
In den letzten Jahren wurden enorme Schritte im Bereich Technologien und Froschung gemacht, und der Trend ist stets positiv. In unseren Augen wird aber, in weniger Jahren in der Welt des Hausautomation, die Kluft zwischen Analog und Digital überwinden.\\
Im verlauf der Arbeit wurde klar das der Weg zur digitalisierung die richtige Entscheidung war. Das digitalen Welt bring mit sich unzählige Vorteile die im zukünftige Hausautomation nicht mehr wegzudenken sind. So ist zum Beispeil die Interaktion mit den \gls{turklingelanlage} über ein Smartphone oder ein Tablet ein erhebliche Vorteil im gegensatz zur analoge Systeme. Weiterhin ermöglicht den digitales System der Zugriff auf die Videoübertragung von Aussen, was mit analoge Technologie bis jetzt kaum realisierbar war.\\
   

    



  
